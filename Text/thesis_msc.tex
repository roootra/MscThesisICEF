% !TeX document-id = {e267d368-ee0e-4347-96b8-d2202efb2395}
% !BIB TS-program = biber
% !TeX encoding = UTF-8
% !TeX spellcheck = en_GB
\documentclass[12pt, a4paper]{extarticle}

%\usepackage{showframe}

\usepackage{cmap}					% поиск в PDF
%\usepackage{mathtext} 				% русские буквы в формулах
%\usepackage[T2A]{fontenc}			% кодировка
\usepackage[utf8]{inputenc}			% кодировка исходного текста
\usepackage[english]{babel}			% локализация и переносы
%\usepackage[urw-garamond]{mathdesign}
%\usepackage{indentfirst}
\usepackage{setspace}
\frenchspacing % добаляет большой пробел между предложениями
%\onehalfspacing
\linespread{1.5}
%\usepackage{helvet}
%\usepackage{newcent}
%\usepackage{lmodern}
\usepackage{calligra}
\DeclareFontFamily{OT1}{pzc}{}
\DeclareFontShape{OT1}{pzc}{m}{it}{<-> s * [1.10] pzcmi7t}{}
\DeclareMathAlphabet{\mathpzc}{OT1}{pzc}{m}{it}

\DeclareUnicodeCharacter{0301}{} %workaround

\usepackage{gensymb}
\usepackage{textcomp}
\usepackage{booktabs}
\usepackage{makecell}
\usepackage{pdflscape}
\usepackage{tabularx}

%БЛОК СООТВЕТСТВИЯ ПРАВИЛАМ
%\renewcommand{\rmdefault}{ftm}
\usepackage{times}

%without babel
\renewcommand{\contentsname}{CONTENTS}
\renewcommand{\refname}{REFERENCES}

\makeatletter
% запрещаем переносы в названиях секций
%\renewcommand{\section}{\@startsection{section}{1}{0pt}%
%	{-3.5ex plus -1ex minus -.2ex}%
%	{2.3ex plus .2ex}%
%	{\centering\hyphenpenalty=10000\normalfont\bfseries}}

% chapter
\usepackage{titlesec}
\titleformat{\chapter}{\thispagestyle{myheadings}\centering\hyphenpenalty=10000\normalfont\huge\bfseries}{
	\thechapter. }{0pt}{\large}
\makeatother

%section
%\titleformat{\section}[block]{\bfseries\large\filcenter}{SECTION \thesection.}{.5em}{}
\titleformat{\section}[block]{\bfseries\filcenter}{}{.5em}{}

% subsection
\titleformat{\subsection}[block]{\bfseries\filcenter}{\thesubsection.}{.5em}{}

%subsubsection
\titleformat{\subsubsection}[block]{\bfseries\filcenter}{\thesubsubsection.}{.5em}{}

%math font size (textfont-disp-script-scriptscript)
\makeatletter
\DeclareMathSizes{\f@size}{12}{10}{9}
\makeatother

%to show current font size in log (14pt = 14.4pt)
\makeatletter
\show\f@size
\makeatother

%toc depth
\setcounter{tocdepth}{2}

%TOC section cap
%\addto\captionsenglish{
%	\renewcommand{\contentsname}{\hfill\bfseriesCONTENTS\hfill} 
%}
%\makeatletter
%\renewcommand{\tableofcontents}{%
%	\vspace*{-4mm}% reduce space before
%    \noindent{\contentsname}%
%	\vspace{2mm}% space between title and rule
%	\hrule
%	\vspace{2mm}% space below rule
%	\@starttoc{toc}}
%\makeatother

%bib section cap
%\DefineBibliographyStrings{english}{%
%	bibliography = {BIBLIOGRAPHY},
%	references = {REFERENCES},
%}


% КОНЕЦ БЛОКА СООТВЕТСТВИЯ ПРАВИЛАМ




%работа с библиографией
%style=authoryear
%sorting=nyt,
\usepackage[bibstyle=gost-authoryear-min,
			citestyle=authoryear,
			maxcitenames=1,
			backend=biber]{biblatex}
\addbibresource{../Sources/lit.bib}
%Убрать тире из библиографического списка
%\renewcommand*{\newblockpunct}{\addperiod\space\bibsentence}.:
%Для возврата к прежнему значению (установленному по умолчанию):
%\renewcommand*{\newblockpunct}{\addperiod\addnbspace\textemdash\space\bibsentence}.



%Дополнительные пакеты
\usepackage{amsmath,amsfonts,amssymb,amsthm,mathtools} %математика
\usepackage{indentfirst} %отступ первого абзаца
\usepackage{graphicx} %вставка рисунков
\usepackage{wrapfig} %обтекание рисунков текстом
\usepackage{geometry} %установка полей
\geometry{top=20mm} % >20
\geometry{bottom=30mm} % >20
\geometry{left=35mm} % =35
\geometry{right=20mm} % >15
\usepackage{soul} %модификаторы начертания
\usepackage{csquotes}
%\usepackage{titling}
\usepackage{enumitem}
\usepackage[toc,page]{appendix}
\usepackage{tocloft}
\usepackage{siunitx}
\usepackage{multicol}

\renewcommand{\appendixtocname}{Appendix}
\renewcommand{\appendixname}{Appendix}
\renewcommand{\appendixpagename}{Appendix}
\renewcommand{\cftsecleader}{\cftdotfill{\cftdotsep}}

\makeatletter
\newcommand{\mathleft}{\@fleqntrue\@mathmargin0pt}
\newcommand{\mathcenter}{\@fleqnfalse}
\makeatother

\usepackage{float}
\usepackage{hyperref}
\usepackage[usenames,dvipsnames,svgnames,table,rgb]{xcolor}
\hypersetup{				% Гиперссылки
	unicode=true,           % русские буквы в раздела PDF
	pdftitle={Determining Exchange Rate Pass-through in Russia},   % Заголовок
	pdfauthor={Artur N. Zmanovskii},      % Автор
	pdfsubject={Macroeconomerics, Macroeconomics},      % Тема
	pdfcreator={}, % Создатель
	pdfproducer={}, % Производитель
	pdfkeywords={Pass-through} {ERPT} {PERR}, % Ключевые слова
	colorlinks=true,       	% false: ссылки в рамках; true: цветные ссылки
	linkcolor=black,          % внутренние ссылки
	citecolor=black,        % на библиографию
	filecolor=black,      % на файлы
	urlcolor=black           % на URL
}

%колонтитулы
\usepackage{fancyhdr} %загрузим пакет
\pagestyle{fancyplain} %применим колонтитул
\fancyhead{} %очистим header на всякий случай
%\fancyhead[LE,RO]{\thepage} %номер страницы слева сверху на четных и справа на нечетных
%\fancyhead[CO]{текст-центр-нечетные}
%\fancyhead[LO]{текст-слева-нечетные} 
%\fancyhead[CE]{текст-центр-четные} 
\fancyfoot{}
%\fancyfoot[RE,RO]{\thepage}
\rfoot{\thepage}
\renewcommand{\headrulewidth}{0pt}
\setcounter{page}{2}

%шапка
\title{{\large DETERMINING EXCHANGE RATE PASS-THROUGH IN RUSSIA}}
\author{Artur N. Zmanovskii\footnote{National Research University `Higher School of Economics'}}
\date{}

%\renewcommand{\appendixname}{Appendix}

\begin{document}
\setcounter{page}{2}
\begin{abstract}
	\textit{tbd}
\end{abstract}
\newpage

\tableofcontents
%\maketitle
\linespread{1.5}

\newpage
%\tableofcontents
%\newpage
\linespread{1.5}

\section*{SECTION 1. INTRODUCTION}
\addcontentsline{toc}{section}{SECTION 1. INTRODUCTION}

\newpage
\section*{SECTION 2. LITERATURE REVIEW}
\setcounter{section}{2}
\addcontentsline{toc}{section}{SECTION 2. DATA}
The goal of pioneering works in exchange rate pass through estimation area was mainly in determining industry-specific effects in specific economies: among others, (\cite{Schembri1985}) examines Canadian exports, (\cite{Menon1992, Menon1993}) --- Australian exports and Imports of Motor Vehicles, (\cite{Khosla1991, Athukorala1994}) --- Japanese exports, (\cite{Cowling1989} --- UK and West German car market, \cite{Athukorala1991} --- Korean exports, \cite{Baldwin1988, Feenstra1989, Hooper1989}) --- US imports. These papers show that there is a heterogeneity in pass-through across industries as well as countries though challenging data measurement errors and model misspecifications. A huge contribution to review these attempts is made in (\cite{Menon1993, Menon1995}).

Looking for exchange rate pass-through for whole economies, \autocite{Khosla1989} estimate shock-independent ERPT to export prices for 23 countries using calculated quarterly nominal effective exchange rate for each economy and fitting OLS regressions. They find that pass-through effect varies drastically across countries: for developed economies this value is high, meanwhile developing ones experience low pass-through.

A more advanced methods are used in (\cite{Kim1990}) --- author examines pass-through to US import prices and influence of exchange rate to mark-up using a model with time-varying parameters. It is shown that a mark-up negatively correlates with US dollar exchange rate, though a direct effect of the latter to prices fell from 1980s. 

In (\cite{Deravi1995}) a vector autoregression (VAR) is applied to fit US broad money aggregate, dollar exchange rate and consumer price index (CPI) with a main emphasis on monetary supply shock. Via causality test It is underlined that there is a significant causality effect of broad money to other macrovariables. Variance analysis suggests the effects to CPI from innovations to other two variables are nearly equal after four years.
 
(\cite{Kim1998}) employs vector error-correction model (VECM) in order to study pass-through to US import prices. This paper reveals a significant negative effect of US exchange rate appreciation to producer price index (PPI) and conducts causality test for this dependency, which confirms an influence of exchange rate. Moreover, author argues that previous works were using inefficient methods to examine ERPT. 

%Taylor2000
In his renown paper, \textcite{Taylor2000} provides strong theoretical framework for understanding exchange rate pass-through nature. The author simulates three-equation model of individual and aggregate prices and output and shows that when the inflation is low, pricing power of firms declines as well leading to lower pass-through. Hence, if a producer wants to raise or lower their individual price due to change in costs or, equivalently, exchange rate, he or she would expect other firms stay on the remaining price level due to low inflation.
 
%DSGE-like papers
Another approach of examining exchange rate pass-through is contained in literature based on general equilibrium models, although there are few ones specially structured for studies in this particular field. Mainly based on purely statistical approach, this particular paper refers only to several works of this kind, leaving the rest to the reader.

One of the works is \cite{Adolfson2001}, where author examines optimal policy of monetary authority under different completeness of pass-through. The main consequence of this study is that the lower pass-through is, the less important nominal economy is, as interest rate response to shocks from outside is lower and exchange rate fluctuations are higher.

The seminal paper in this field is \cite{Obstfeld2002}. It does not directly touch the pass-through problem, however, it is a starting point for many papers in this field. In the paper, a cooperation of monetary authorities in a two-country model is examined. The main result of this paper is that even if monetary authorities do not coordinate with each other, benefits from macroeconomic stabilization can outweigh lack of coordination, and coordination under fixed exchange rate is more preferred than one under the floating rate.

Looking for effects of exchange rate volatility, (\cite{Devereux2002}) develop a multi-economy new-Keynesian general equilibrium model based on the model from aforementioned paper. Authors show that fluctuations in nominal exchange rate appear to compensate pass-through to prices nominated in local currencies. It is argued that even if there is a little volatility in fundamental macroeconomic variables, fluctuations of exchange rate may be quite high. This model lacks empirical research though, constrained only by simulations with different parametrisation.

Basing on the same foundations, an attempt to make an empirical research based on DSGE model is done in (\cite{Smets2002}), where Euro area data is used to calibrate a model and estimate exchange rate pass-through in an economy with optimal monetary policy. As a result, authors claim that under an assumption of presence of import price stickiness in the economy, its effect is similar as stickiness of domestic prices.

%Pretty not DSGE
\textcite{Gagnon2004} use Monte-Carlo approach with multi-equation model to show that there was a decline in exchange rate pass-through since 80s due to inflation stabilisation policy conducted by many central banks across the world. To find more evidence, authors fit an OLS regression with lags of exchange rate summed with foreign CPI for two subsamples individually chosen for 20 countries. Additionally, they estimate interest rate rule coefficients in order to find changes in monetary policy. Finally, authors argue that the hypothesis is confirmed.

%Current Empirics
A new wave in studying exchange rate pass-through --- use of structural vector autoregressions (SVAR) --- starts from (\cite{Hahn2003}) for Euro area macro data from 1971 to 2002. In this remarkable work, a recursive (also known as \textit{Cholesky}) identification scheme is used in order to recover macroeconomic shocks to PPI and HICP from other different macroeconomic variables (oil price, interest rate, output gap and non-oil import prices). To address statement about pass-through decline in (\cite{Gagnon2004}), author conducts a robustness test and finds out that there was no significant change in pass-through effect for the Euro area.

The same conclusion about decline, among other ones, is made in (\cite{Campa2005}). Searching for the pass-through effect to import prices, authors examine data for 23 countries and assert that the pass-through effect is incomplete for all countries in the short run and for overwhelming majority of them in the long run.

(\cite{CaZorzi2007}) and (\cite{McCarthy2007}) papers resemble previously cited (\cite{Hahn2003}). The first work studies data for 12 developing economies and employs recursive SVAR to estimate shock-dependent ERPT; authors find that pass-through effect fades down to the distribution chain and argue that when inflation in a developing economy is low, ERPT is comparable to one of developed countries.

On the opposite side is (\cite{McCarthy2007}) work, where data for nine developed countries are examined applying Cholesky identification scheme to VAR. Author states that pass-through in developed economies is quite low, and inflation in the US is mainly driven by oil shocks, producer price shocks and internal CPI shocks.

In (\cite{Shambaugh2008}) paper author uses long run restrictions for SVAR in order to identify link between exchange rate and CPI together with import prices. Author uses data for 16 countries for the time frame from 1973 to 1999 and obtains supportive evidence that low inflation declines pass-through --- for some countries, CPI growth rate does not respond to exchange rate shocks in the same magnitude as producer price index growth rate.

Data's granularity higher than quarterly is not usually found in the studies, although (\cite{Amstad2010}) observe monthly Swiss CPI and NEER from 1993 to 2008. This work employs event study approach to estimate an effect of monthly import price time series release to ERPT. Author underlines that this method is more suitable for policymakers due to possibility of using the most current data and does not rely on VAR restrictions, which may be controversial. The criticism of SVAR is quite questionable in this light, since the monthly data does not impair a possibility of proper identification of shocks, while the benefits of shock-dependent ERPT are higher for monetary and macroprudential authorities.

An innovative identification method is introduced by \textcite{An2012} --- author employs sign identification scheme in order to obtain price-to-exchange rate ratio (\textit{PERR}, shock-dependent exchange rate pass-through), which will be described in the following Section. Author fits the model for eight developed economies and claims that for the most cases pass-through is incomplete. Another conclusion is that pass-through is higher for small-sized economies with more volatile monetary policy.

The work of \textcite{Delatte2012} is devoted to determination of pass-through asymmetry for four countries (Germany, Japan, UK, US) from 1980 to 2009. An ARDL with nominal exchange rate changes divided into two variables (with positive and negative increments) is estimated to determine both short-run and long-run asymmetric ERPT. Author argues that pass-through is smaller during local currency appreciations.

(\cite{BrunAguerre2012}) paper's aim is to find what drives ERPT to import prices. Authors use both ECM and panel fixed effects (FE) model to catch time- and country-specific effects for 37 countries on 1980--2009 period; again, pass-through asymmetry is considered. The conclusion is that there is no evidence of pass-through declining for both developed and emerging economies, although domestic tariffs and import-to-export ratio matter.

Monthly data of Taiwanese economy under deflation are examined in (\cite{Lin2012}). In this work, a two- and three-regime threshold autoregression (TAR) models are fit to find non-linearities in pass-through relation. It is argued that pass-through declines only when inflation is close to zero, and the link of ERPT and inflation is V-shaped. With this non-trivial result, high rates of deflation are unpleasant for an economy additionally from the side of exchange rate pass-through.

Another work observing Asian economy is (\cite{Jiang2013}). Authors estimate SVARs with custom shock matrix in order to find PERR for China. This method is more flexible than recursive identification scheme as the shock matrix does not necessarily need to be triangular, although application of such scheme is quite situational. Authors conclude that PERRs are incomplete, which is usual for the literature in this field.

(\cite{Yamada2013}) paper is devoted to study exchange rate regime effect to inflation among developing and emerging economies. Author fit treatment effects model with propensity score matching based on GDP and geographical characteristics in order to calculate between inflation targeting regime and other ones. The conclusion is that inflation targeting exchange rate regime performs at least not worse than fixed regime in terms of inflation lowering.

Multi-currency study for 17 countries of Euro area is done by \textcite{Bandt2014} to estimate effect of exchange rate fluctuations to import prices for multiple trade partners. Currencies chosen are US dollar, UK pound-sterling and Chinese Renminbi. Authors estimate FE model in order to calculate ERPT and find out that in the short run pass-through is incomplete, but its completeness is confirmed fore the long run.

In order to look for the changes in pass-through after 2008 financial crisis, \textcite{Jasova2016} estimate 6-year rolling ERPT for both developing (11) and advanced (22 countries) economies completing their study by fitting two-way FE model. Authors assert that pass-through declined during financial crisis for developing economies, meanwhile ERPT of developed countries remained on a relatively stable level.

In (\cite{Comunale2017}) paper data for four Euro area countries --- France, Germany, Italy and Spain --- are studied to find both ERPT and PERR under the zero lower bound (ZLB) environment. Instead of short-term interest rate, authors make use of calculated \textit{shadow interest rates} and estimate Bayesian VAR with sign and zero restrictions. The results of the study are that pass-through is high and volatile to import prices and, in general, is dependent on shocks evolving. Moreover, authors state that the process of choosing identification scheme is quite sophisticated, and the identified shocks are true only conditional on the scheme involved.

Both FE model and sign and zero restricted SVAR are estimated in (\cite{Forbes2017}), where authors try to analyse time- and country-specific differences in pass-through on the sample of 26 countries. It is argued that structural variables, like the first two statistical moments of inflation and exchange rate are important for time and country effect explanation, while structural shocks are crucial for explanation of macro-variable variation in time.

A quite remarkable paper of the same collective is (\cite{Forbes2018}). Authors study UK economy pass-through before and during Brexit using SVAR model with sign and zero restrictions. The study shows that pound-sterling's depreciation periods during 2008 financial crisis and Brexit have different ground and discrepancy in inflation rates are caused by different shocks affecting the economy. Authors admit that set up in this fashion, a model cannot capture all the complexity of pass-through nature, although identification of shocks can help to improve relevant policy by monetary authority.

Another work employing the same method is (\cite{An2020}), though \textit{narrative} sign restrictions (simply put, signs dictated by historical events) are added. The main drivers of Japanese pass-through are examined in this study. Authors argue that narrative sign approach is more promising in terms of shock identification procedure.

Time-varying ERPT is examined in (\cite{LeivaLeon2019}), where authors estimate time-varying parameters (TVP) dynamic factor model and SVAR with sign restrictions for Euro area. A TVP approach is quite innovative in pass-through literature, as it is highly likely to solve problems of non-linear ERPT estimation. The paper's conclusion is that inflation is mostly driven by exogenous exchange rate shocks, though core inflation are less exposed to these fluctuations.

\textcite{Colavecchio2019} use local projection method in order to capture non-linear pass-through effects for the 19 countries in Euro area. Plainly speaking, local projections are \textit{h}-ahead forecasts on the basis of current data. The results show that there is no complete pass-through for all the countries neither after a one year nor two years. Authors also find there is a sign and exchange rate shock size non-linearity for some countries.

The recent work of \textcite{Comunale2020} is devoted to a comparison of Bayesian SVARs and DSGEs for the purposes of PERR estimation. This particular work is important in the sense that SVAR and DSGE models can give controversial results; hence, a policymaker needs to distinguish an appropriate aims for both setups. Author finds out that just after a shock PERR's are identical for both models, although in the long-run estimates from DSGE are higher due to endogenous response of macrovariables. 

DSGE is also employed in (\cite{GarciaCicco2020}), where comparison of shock-independent ERPT's and PERR's is done on Chilean data. It is argued throughout the work that pass-through conditional on shocks gives a full picture of macroeconomic variables' relations and that DSGE models are helpful to generate prudent monetary policy. 

In their latest work, \textcite{Forbes2020} estimate both SVAR and FE model for a set of 26 countries in order to review \textit{''shocks vs. structure''} dilemma. Authors claim that both structural characteristics and shocks are important for better understanding pass-through. Also they find an evidence that monetary shocks are associated with large PERR, which made a big contribution to price fluctuations in advanced economies that are not close to the lower bound.

All in all, there has been a shift in the literature from industry-specific studies to understanding of shock importance during exchange rate pass-through estimation. This drift is dictated not only by an evolution of methodology (this point is prevalent though), but data availability and, what is the most important, a switch to macroprudential policy. The intervention of vector autoregressions and Bayesian estimation techniques, especially sign restrictions, have given, in some sense, the second breath to the research. On the other hand, further development of DSGE models and acceptance of them by central banks globally gave an idea of how shock-dependent ERPT should look like for each country. The idea of SVAR being guided by DSGE (at least for a signs) is given in (\cite{Ortega2020}), which is a brilliant review on the topic of exchange rate pass-through estimation with a focus on Euro area and the US.



%%%
\section*{SECTION 3. METHODS}
\setcounter{section}{3}
\addcontentsline{toc}{section}{SECTION 3. METHODS}


\section*{SECTION 4. DATA}

\setcounter{section}{2}
\addcontentsline{toc}{section}{SECTION 4. DATA}


\section*{SECTION 5. RESULTS}
\setcounter{section}{4}
\setcounter{subsection}{0}
\addcontentsline{toc}{section}{SECTION 5. RESULTS}

\clearpage

\section*{DISCUSSION}
\addcontentsline{toc}{section}{DISCUSSION}


%%%%%%%%%%%%%%%%%%%%
\renewcommand*{\newblockpunct}{\addperiod\space\bibsentence}
\newpage
\linespread{1.3}

\printbibliography
\addcontentsline{toc}{section}{References}
\newpage
%counters in captions
\setcounter{figure}{1}
\setcounter{table}{1}
\makeatletter
\renewcommand*{\thetable}{\alph{table}}
\renewcommand*{\thefigure}{\alph{figure}}
\let\c@table\c@figure
\makeatother 

\normalsize
\linespread{0.5}
\section*{Appendix A: }
\label{app:a}

~\\
\clearpage
\newpage
\pagebreak

\end{document}
